\documentclass{TDP005mall}



\newcommand{\version}{Version 0.1}
\author{Hadi Ansari, \url{hadan326@student.liu.se}\\
  Nils Bark, \url{nilba048@student.liu.se}}
\title{Kravspecifikation}
\date{2020-11-16}
\rhead{Hadi Ansari\\
Nils Bark}



\begin{document}
\projectpage
\section{Revisionshistorik}
\begin{table}[!h]
\begin{tabularx}{\linewidth}{|l|X|l|}
\hline
Ver. & Revisionsbeskrivning & Datum \\\hline
0.1 & Första utkastet & 2020-11-17 \\\hline

\end{tabularx}
\end{table}


\section{Spelidé}
Spelet utspelar sig i en 2D-miljö från ett sidoperspektiv. Spelaren kan röra sig i åtta olika riktningar och skjuta skott mot de ankommande fienderna. Skotten som spelaren skjuter gör skada till fienderna den träffar och kommer till slut förstöra den. När en fiende förstörs får spelaren en visst antal poäng. Det finns även en chans att en förstörd fiende släpper ifrån sig en ``Power-up'' som ger spelaren olika fördelar. Det kommer även då och då att dyka upp Power-ups från högra sidan av skärmen. Fienderna kommer att närma sig från höger om skärmen och röra sig mot vänster. Fiendertyperna kan även ha unika rörelsemönster när de färdas över skärmen. Fienderna gör skada till spelaren antigen genom att vidröra den eller genom att träffa spelaren med sina skott (om de skjuter skott).

\section{Målgrupp}
Målgruppen är alla som tycker om utmanande skjut-spel i denna klassiska arkad-stil.

\section{Spelupplevelse}
Det roliga med spelet fås från en kombination av den ständigt ökande utmaningen och poänginsamlingen. Spelet kan spelas om och om igen för att få ett så bra poängrekord som möjligt.

\section{Spelmekanik}
Spelaren kontrolleras av W, A, S, D eller piltangenterna tillsammans med mellanslag.

\section{Regler}


\end{document}
