\documentclass{TDP005mall}



\newcommand{\version}{Version 0.1}
\author{Hadi Ansari, \url{hadan326@student.liu.se}\\
  Nils Bark, \url{nilba048@student.liu.se}}
\title{Kravspecifikation}
\date{2020-11-22}
\rhead{Hadi Ansari\\
Nils Bark}



\begin{document}
\projectpage
\tableofcontents
\thispagestyle{empty}
\cleardoublepage

\section{Revisionshistorik}
\begin{table}[!h]
\begin{tabularx}{\linewidth}{|l|X|l|}
\hline
Ver. & Revisionsbeskrivning & Datum \\\hline
0.1 & Första utkastet & 2020-11-17 \\\hline
1.0 & Komplettering efter feed-back& 2020-11-22 \\\hline

\end{tabularx}
\end{table}


\section{Spelidé}
Spelet utspelar sig i en 2D-miljö från ett sidoperspektiv.
Spelaren kan röra sig i åtta olika
riktningar och skjuta skott mot de ankommande fienderna. Skotten som spelaren skjuter gör
skada till fienderna om den träffar och kommer till slut förstöra den. När en fiende förstörs
får spelaren en visst antal poäng. Det finns även en chans att en förstörd fiende släpper
ifrån sig en ``Power-up'' som ger spelaren olika fördelar. Det kommer även då och då att
dyka upp Power-ups från högra sidan av skärmen. Fienderna kommer att närma sig från höger
om skärmen och röra sig mot vänster. Fiendertyperna kan även ha unika rörelsemönster när
de färdas över skärmen. Fienderna gör skada till spelaren antigen genom att vidröra den
eller genom att träffa spelaren med sina skott (om de skjuter skott). Spelaren har ett
antal liv som förloras allt eftersom den har blivit träffad av fienderna/deras skott.
När spelarens liv är noll då spelet är slut.

\section{Målgrupp}
Målgruppen är alla som tycker om utmanande skjut-spel i denna klassiska arkad-stil.

\section{Spelupplevelse}
Det roliga med spelet fås från en kombination av den ständigt ökande utmaningen och poänginsamlingen. Spelet kan spelas om och om igen för att få ett så bra poängrekord som möjligt.

\section{Spelmekanik}
Spelaren styrs av W, A, S, D eller piltangenterna. Spelaren skjuter med mellanslag.
\begin{table}[!h]
\begin{tabularx}{\linewidth}{|X|X|}
\hline
Tangentknapp & Åtgärd\\\hline
W & Flytta spelaren uppåt\\\hline
S & Flytta spelaren nedå\\\hline
A & Flytta spelaren vänster\\\hline
D & Flytta spelaren höger\\\hline
Mellanslag & Skjuta skott\\\hline

\end{tabularx}
\end{table}

\section{Regler}
Nedan beskrivs regler för varje specefik spelobjekt samt poängsamlingen. 
\subsection{Spelplan}
\begin{itemize}
\item Spelplanen har en fixerad storlek.
\item Spelaren kan endast befinna sig inom spelplanen medan fienderna kan komma in i spelplanen från sidorna. 
\item Objekt som åker ut ur spelplanen förstörs.
\end{itemize}

\subsection{Spelare}
\includegraphics[scale=0.2]{Images/Player.png}
\begin{itemize}
\item Spelaren kan röra på sig i åtta riktningar.
\item Spelaren kan endast skjuta skott framåt och horisontellt som skadar fienderna när de kolliderar.
\item Spelarens skott minskar fienders liv med 1.
\item Spelaren har en bashastighet som avgör hur snabbt den kan skjuta.
\item Spelaren kan plocka upp Power-ups genom att kolliderar med dem.
\item Spelaren inte kan plocka upp Power-upps genom att förstöra dem med sin skott.
\item Spelaren har tre liv och kan inte gå över detta.
\item Spelaren kan få liv genom att plocka upp heal Power-ups.
\item Spelaren förlorar ett liv när den kolliderar med skot.
\item Spelaren förlorar ett eller två liv när den kolliderar med fiender (beroende på fiendetypen).
\item När spelaren ta skada (kollision med fiende objekt eller deras skott) blir den odödlig i tre sekunder.
\item När spelarens liv är noll förlorar man.


\end{itemize}

\subsection{Fiende}
Fiender i spelet kan inte förstöra varandra med skott eller kollision. De inte heller kolliderar med Power-ups eller förstöra de med sitt skott. Nedan kommer allmänna regler som har tänkts följas i spelet.

\begin{itemize}
\item Fienderna kan röra på sig.
\item Fienderna rör sig enligt förbestämda mönster (beroende på typen) över skärmen.
\item Fienderna skjuter skott enligt en bashastighet som spelaren.
\item Fiender kan skada spelaren genom att kollidera med den.
\item Fiender har också ett antal liv och när de deras liv når noll dör de.
\item Fiender kommer från höger sidan om skärmen och de rör sig åt vänster tills de når slutet av sin bana.
\end{itemize}

Olika typer av fiender i spelet består av följande.

\subsubsection*{Fiende typ1}
\includegraphics[scale=0.15]{Images/Enemy1.png}
\begin{itemize}
\item Den har 1 liv.
\item Den förstörs när den kolliderar med spelaren eller när den träffar spelarens skott.
\item När den kolliderar med spelaren minskar spelarens liv med 1.
\item Dens skott minskar spelarens liv med 1 när den kolliderar med den.
\item Den rör sig åt vänster enligt en bestämd vågrörelse (se figur 1).
\item Den har röreslehastighet 3.
\item Den kan endast skjuta skott framåt och horisontellt.
\item Den sjkuter skott när den är i samma höjd med spelaren.
\end{itemize}
\begin{figure}[h!]
  \centering
  \includegraphics[scale=0.4]{Images/Enemy1-movement.png}
  \label{Bild 1}
  \caption{Hur fiende typ 1 rör sig åt vänster}
\end{figure}




\subsubsection*{Fiende typ2}
\includegraphics[scale=0.2]{Images/Enemy2.png}
\begin{itemize}
\item Den har 2 liv.
\item Den förstörs (oavsett om den har ett eller två liv) när den kolliderar med spelaren.
\item Den förstårs när den träffar spelarens skott två gånger.
\item När den kolliderar med spelaren minskar spelarens liv med 2.
\item Dens skott minskar spelarens liv med 1.
\item Den rör sig åt vänster enligt en linjärt mönster (se bild).
\item Den har röreslehastighet 2.
\item Den sjkuter när den är i samma höjd med spelaren.
\end{itemize}
\begin{figure}[h!]
  \centering
  \includegraphics[scale=0.4]{Images/Enemy2-movement.png}
  \label{Bild 2}
  \caption{Hur fiende typ 2 rör sig åt vänster}
\end{figure}

\newpage
\subsubsection*{Bomber}
\includegraphics[scale=0.08]{Images/Bomb.png}\\
\begin{itemize}
\item De dycker upp från höger och närmer sig till vänster sidan av skärmen med rörelsehastighet 1.
\item Bomber minksar spelarens liv med 1 när de kolliderar med spelaren.
\item De förstörs när de kolliderar med spelaren eller när de kolliderar med spelarens skott.

\end{itemize}

\subsection{Skott}
\begin{itemize}
\item Skotten kan skjutas av båda spelaren och fiende-typerna 1 och 2.
\item Skotten har hastighet 4 (snabbare än alla föremål i spelet).
\item De skadar motståndarens och minskar dens antal liv med 1.
\end{itemize}

\subsection{Power-up \& föremål}
Power-ups kan ge spelaren diverse positiva effekter. De kommer i spelplanen från höger och närmer sig till vänser sidan. Ur spelarens perspektiv rör de på sig med hastighet 1 och plockas upp när dem kolliderar med spelaren.

\begin{itemize}
%% \item Power-ups kan släppas ifrån fiender när spelaren förstör dem.
%% \item Power-ups kan också komma in i spelplanen från sidorna.
\item Power-ups kan ge spelaren diverse positiva effekter.
%% \item Livlådor kan dyka upp antingen från förstörda fiender eller från kanten av skärmen precis som power-ups. De ger spelaren ett liv vid upplockning.
\end{itemize}

\subsubsection*{Heal}
\begin{itemize}
\item Heal Power-up ökar spelarens liv med 1.
\item Spelaren kan inte få mer än tre liv.
  
\end{itemize}

\subsubsection*{Triple-shot}
\begin{itemize}
\item Triple-shot Power-up gör så att spelaren kan skjuta tre skott i varje skjutning.
\item Denna effekt försvinner efter 15 sekunder.
\item Spelaren kan inte få två Triple-shot power-ups samtidigt (så att det blir 30 sekunder triple-shot effekt).
\end{itemize}

\subsubsection*{Shield}
\begin{itemize}
\item Sheild Power-up gör spelaren odödlig i 10 sekunder.
\item Spelaren kan inte få två Sheild power-ups samtidigt (så att det blir 20 sekunder odödlighet)
\end{itemize}


\subsection{Poäng}
\begin{itemize}
\item Poäng kan samlas genom att förstöra fienderna.
\item Spelaren får 50 poöng när den förstör en bomb med sitt skott.
\item Spelaren får 100 poäng när den förstör ett fiende typ 1 med sitt skott.
\item Spelaren får 150 poäng när den förstör ett fiende typ 2 med sitt skott.
\item Spelaren får 300 poäng om den plockar upp en Heal Power-up när den har redan fullt liv  
\item Uppsamling av minst 70\% av den totala poäng i spelet när spelaren har klarat av banan ger en tre-stjärna segern.
\item Uppsamling av minst 50\% av den totala poäng i spelet när spelaren har klarat av banan ger en två-stjärna segern.
\item Avklarat bana med mindre än 50\% av den totala poäng i spelen resulterar en segern med en sjtärna.

\end{itemize}

\section{Visualisering}
Det här är ett exempel på hur spelet kan se ut. Observera att spelaren är den vänstra flygplanet som flyger åt vänster. Alla fiender och Power-upps rör sig åt höger enligt sitt mönster. Bomber och Power-ups rör sig också (långsammare än fiende flygplanen) för att ger ett känsla att spelaren flyger åt höger hela tiden.

\begin{figure}[h!]
  \centering
  \includegraphics[scale=0.35]{Images/Game.png}
  \label{Bild 3}
  \caption{Samtliga spelobjekt (förutom spelaren) kommer in i skärmen från höger kanten och förljer sitt mönster tills de når vänstra kanten av skärmen}
\end{figure}


\section{Krav}
Nedan kommer ett antal krav som spelet i slutändan ska uppfylla. Ska-krav är högst prioriterad medan bör-krav ska uppfyllas om det finns gott om tid och kunskap.

\subsection{Ska-krav}
\begin{enumerate}
\item Spelplanen ska ha en fixerad storlek.
\item Spelet utspelar sig i en 2D-miljö sedd från sidoperspektiv.
\item Spelet ska ha en förstasida där en ny runda kan startas, spelet kan avslutas, och eventuella andra sidor kan visas.
\item Spelet ska ha en hjälp-sida där regler och kontroller förklaras.
\item Spelaren ska kunna röra på sig i åtta riktningar som endast begränsas av spelplanens kanter.
\item Spelaren styrs via tangentbordet.
\item Spelaren kan skjuta skott horisontellt åt höger.
\item Spelaren kan förstöra fiender med sina skott (1 skott för att förstöra fiende-typ 1 och 2 skott för att förståra fiende typ2).
\item Spelaren kan dö genom att kollidera med fiender eller dess skott ett antal gånger (beroende på skadan som respektiv fiende-objekt orsakar (se specifikationerna ovan för fiender)).
\item Efter kollison ska fiender förstöras.
\item Spelaren ska samla poäng när den förstör ett fiende.
\item Spelaren ska samla poäng när den plockar upp en Heal Power-up när den har fullt liv.
\item Spelarens liv och poäng ska konstant visas på skärmen under rundans gång.
\item Spelaren ska bli odödlig när den tar skada i tre sekunder.
\item När spelaren dör ska rundan avslutas och spelaren ges möjlighet att spela igen.
\item Det ska finnas minst två typer av fiender som båda kan röra på sig.
\item Vissa fiender ska skjuta skott (typ1 och typ2).
\item Fiender ska skada spelaren antingen genom att träffa den med skott eller genom att kollidera med den.
\item Fiender ska böjra skjuta horisontellt när de befinner sig i samma höjd med spelaren.1§
\item Fiender ska ha en chans att släppa ifrån sig power-ups när de förstörs av spelaren.
  %\item Spelet ska öka i svårhetsgrad ju längre spelaren lever. % Hur ska detta ske? Fler fiender? Färre power-ups?
\item Det ska finnas ett nivå (Level 1) i spelet.
\item Det ska kunna läggas till flera nivåer med samma spelobjekten.
\item Det ska kunna finnas flera fiender av samma typ på spelplanen samtidigt.
\item Spelaren ska kontrollera en spelarkaraktär.
\item Fienderna ska röra på sig enligt ett mönster i spelbanan.
\item Spelaren kan plocka upp Power-ups genom att kollidera med dem.
\end{enumerate}

\subsection{Bör-krav}
\begin{enumerate}
\item Power-ups ska dras mot spelaren då den kommer i närheten av dem
\item Antalet intjänade poäng ska sparas när spelaren till slut förlorar och rundan är avslutad.
\item Spelet ska ha en sida där intjänade poäng från tidigare rundor visas i en topplista.
\item Spelet ska ha en sida där användaren kan ändra vilka tangenter som gör vad.
\item Fiender av samma typ ska kunna ha olika rörelsemönster
%% \item Spelaren ska förlora poäng när den tar skada. % Kanske på andra sätt också?
\item Det ska finnas svårare versioner av varje fiendetyp som introduceras allt eftersom spelaren överlever en längre tid
\item Bakgrunden ska röra på sig för att ge användaren en känsla av hastighet.
\item Det ska finnas fler rörelsemönster som låter fienderna komma in i spelplanen från andra platser än bara högerkanten.

\end{enumerate}

\section{Kravuppfyllelse}
\textit{\textbf{``Spelet ska simulera en värld som innehåller olika typer av objekt. Objekten ska ha olika beteenden och röra sig i världen och agera på olika sätt när de möter andra objekt.''}}


Uppfylls av krav 3, 4, 5, 6, 12, 13, 14\\


\textit{\textbf{``Det måste finnas minst tre olika typer av objekt och det ska finnas flera instanser av minst två av dessa. T.ex ett spelarobjekt och många instanser av två olika slags fiendeobjekt.''}}


Uppfylls av krav 12, 17, 18\\

\textit{\textbf{``Ett beteende som måste finnas med är att figurerna ska röra sig över skärmen. Rörelsen kan följa ett mönster och/eller vara slumpmässig. Minst ett objekt, utöver spelaren ska ha någon typ av rörelse.''}}


Uppfylls av krav 3, 12, 19\\

\textit{\textbf{``En figur ska styras av spelaren, antingen med tangentbordet eller med musen. Du kan även göra ett spel där man spelar två stycken genom att dela på tangentbordet (varje spelare använder olika tangenter). Då styr man var sin figur.''}}


Uppfylls av krav 3, 4\\

\textit{\textbf{``Grafiken ska vara tvådimensionell.''}}


Uppfylls av krav 20\\

\textit{\textbf{``Världen (spelplanen) kan antas vara lika stor som fönstret ''}}


Uppfylls av krav 21\\

\textit{\textbf{``Det ska finnas kollisionshantering, det vill säga, det ska hända olika saker när objekten möter varandra, de ska påverka varandra på något sätt. T.ex kan ett av objekten tas bort, eller så kan objekten förvandlas på något sätt, eller så kan ett nytt objekt skapas. ''}}


Uppfylls av krav 6, 14, 22\\

\textit{\textbf{``Det ska vara enkelt att lägga till eller ändra banor i spelet. Detta kan exempelvis lösas genom att läsa in banor från en fil, eller genom att ha funktioner i programkoden som bygger upp en datastruktur som definierar en bana.''}}


Vi antar att detta uppfylls av de funktioner vi använder för att få en varierande svårighetsgrad genom att låta fiender röra sig på olika sätt och öka svårighetsgrad. Men vi är osäkra på om detta krav kan uppfyllas på rätt sätt i ett spel som inte har klart definerade banor.\\

\textit{\textbf{`` Spelet måste upplevas som ett sammanhängande spel som går att spela! ''}}

Ska uppfyllas av alla ska-krav tillsammans.

\end{document}

