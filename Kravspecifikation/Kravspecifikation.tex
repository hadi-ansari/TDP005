\documentclass{TDP005mall}



\newcommand{\version}{Version 0.1}
\author{Hadi Ansari, \url{hadan326@student.liu.se}\\
  Nils Bark, \url{nilba048@student.liu.se}}
\title{Kravspecifikation}
\date{2020-11-16}
\rhead{Hadi Ansari\\
Nils Bark}



\begin{document}
\projectpage
\section{Revisionshistorik}
\begin{table}[!h]
\begin{tabularx}{\linewidth}{|l|X|l|}
\hline
Ver. & Revisionsbeskrivning & Datum \\\hline
0.1 & Första utkastet & 2020-11-17 \\\hline

\end{tabularx}
\end{table}


\section{Spelidé}
Spelet utspelar sig i en 2D-miljö från ett sidoperspektiv. Spelaren kan röra sig i åtta olika riktningar och skjuta skott mot de ankommande fienderna. Skotten som spelaren skjuter gör skada till fienderna den träffar och kommer till slut förstöra den. När en fiende förstörs får spelaren en visst antal poäng. Det finns även en chans att en förstörd fiende släpper ifrån sig en ``Power-up'' som ger spelaren olika fördelar. Det kommer även då och då att dyka upp Power-ups från högra sidan av skärmen. Fienderna kommer att närma sig från höger om skärmen och röra sig mot vänster. Fiendertyperna kan även ha unika rörelsemönster när de färdas över skärmen. Fienderna gör skada till spelaren antigen genom att vidröra den eller genom att träffa spelaren med sina skott (om de skjuter skott). Spelaren har ett antal liv som förloras allt eftersom den har blivit träffad av fienderna/deras skott. När spelarens liv är noll då spelet är slut.

\section{Målgrupp}
Målgruppen är alla som tycker om utmanande skjut-spel i denna klassiska arkad-stil.

\section{Spelupplevelse}
Det roliga med spelet fås från en kombination av den ständigt ökande utmaningen och poänginsamlingen. Spelet kan spelas om och om igen för att få ett så bra poängrekord som möjligt.

\section{Spelmekanik}
Spelaren kontrolleras av W, A, S, D eller piltangenterna tillsammans med mellanslag.

\section{Regler}
\subsection{Spelplan}
\begin{itemize}
\item Spelplanen har en fixerad storlek.
\item Spelaren kan endast befinna sig inom spelplanen medan fienderna kan komma in i spelplanen från sidorna. 
\item Objekt som åker ut ur spelplanen förstörs.
\end{itemize}

\subsection{Spelare}
\begin{itemize}
\item Spelaren kan röra på sig i åtta riktningar.
\item Spelaren kan skjuta skott som skadar fienderna när de kolliderar.
\item Spelaren kan plocka upp Power-ups genom att kolliderar med dem.
\item Spelaren har ett antal liv.
\item Spelaren kan få liv genom att plocka upp Power-ups.
\item Spelaren förlorar liv när den kolliderar med skott eller fiender.
\item När spelarens liv är noll förlorar man.

\end{itemize}

\subsection{Fiende}
\begin{itemize}
\item Fienderna kan röra på sig.
\item Fiender rör sig enligt förbestämda mönster över skärmen.
\item Fiender kan skjuta skott baserat på en timer (till exempel var tredje sekund).
\item Fiender kan skada spelaren genom att kollidera med den.
\item Fiender har också ett antal liv och när de deras liv når noll dör de.
\end{itemize}

\subsection{Power-up \& föremål}
\begin{itemize}
\item Power-ups kan släppas ifrån fiender när spelaren förstör dem.
\item Power-ups kan också komma in i spelplanen från sidorna.
  \item Power-ups kan ge spelaren diverse positiva effekter.
\end{itemize}

\subsection{Poäng}
\begin{itemize}
\item Poäng kan samlas genom att förstöra fienderna.
\item Svårare fiender ger mer poäng.
\item Man får också ett visst antal poäng för varje sekund man lever.
\end{itemize}


\section{Krav}
\subsection{Ska-krav}
\begin{itemize}
\item Spelaren ska kunna röra på sig i en 2d-värld i åtta riktningar
\item Spelaren styrs av tangentbordet.
\item Spelaren kan skjuta skott.
\item Spelaren kan förstöra fiender med sin skott.
\item Det ska finnas minst två fiender som kan röra på sig
\item Fiender kan skjuta skott.
\item Fiende kan skada spelaren antingen med skott eller kollidering.
\item Spelaren kan dö.
\item Spelaren kan samla poäng när den förstör ett fiende.
\item 
\end{itemize}

\subsection{Bör-krav}
\begin{itemize}
\item 
  
\end{itemize}
\end{document}

